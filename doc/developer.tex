%
\documentclass{article}
\usepackage{color, xcolor, colortbl}
\usepackage{graphicx,epstopdf}
\usepackage{geometry}
\usepackage{amsmath,amssymb,amsthm}
\usepackage{algorithm}
\usepackage{algorithmic}
\usepackage{bm}
\usepackage[caption=false]{subfig}
\usepackage{appendix}
\usepackage{multirow}
\usepackage{braket}
\usepackage{hyperref}
\usepackage[capitalize]{cleveref}
\usepackage[english]{babel}


\newcommand{\bvec}[1]{\mathbf{#1}}

\newcommand{\va}{\bvec{a}}
\newcommand{\vb}{\bvec{b}}
\newcommand{\vc}{\bvec{c}}
\newcommand{\vd}{\bvec{d}}
\newcommand{\ve}{\bvec{e}}
\newcommand{\vf}{\bvec{f}}
\newcommand{\vg}{\bvec{g}}
\newcommand{\vh}{\bvec{h}}
\newcommand{\vi}{\bvec{i}}
\newcommand{\vj}{\bvec{j}}
\newcommand{\vk}{\bvec{k}}
\newcommand{\vl}{\bvec{l}}
\newcommand{\vm}{\bvec{m}}
\newcommand{\vn}{\bvec{n}}
\newcommand{\vo}{\bvec{o}}
\newcommand{\vp}{\bvec{p}}
\newcommand{\vq}{\bvec{q}}
\newcommand{\vr}{\bvec{r}}
\newcommand{\vs}{\bvec{s}}
\newcommand{\vt}{\bvec{t}}
\newcommand{\vu}{\bvec{u}}
\newcommand{\vv}{\bvec{v}}
\newcommand{\vw}{\bvec{w}}
\newcommand{\vx}{\bvec{x}}
\newcommand{\vy}{\bvec{y}}
\newcommand{\vz}{\bvec{z}}

\newcommand{\vF}{\bvec{F}}
\newcommand{\vR}{\bvec{R}}


%\newcommand{\diag}{\mathrm{diag}~}
\renewcommand{\Re}{\mathrm{Re}}
\renewcommand{\Im}{\mathrm{Im}}
\newcommand{\I}{\mathrm{i}}

\newcommand{\Tr}{\mathrm{Tr}}

\newcommand{\mc}[1]{\mathcal{#1}}
\newcommand{\mf}[1]{\mathfrak{#1}}
\newcommand{\wt}[1]{\widetilde{#1}}

\newcommand{\abs}[1]{\left\lvert#1\right\rvert}
\newcommand{\norm}[1]{\left\lVert#1\right\rVert}


\newcommand{\ud}{\,\mathrm{d}}
\newcommand{\Or}{\mathcal{O}}
\newcommand{\EE}{\mathbb{E}}
\newcommand{\NN}{\mathbb{N}}
\newcommand{\RR}{\mathbb{R}}
\newcommand{\CC}{\mathbb{C}}
\newcommand{\ZZ}{\mathbb{Z}}

% For electronic structure
\newcommand{\ext}{\mathrm{ext}}
\newcommand{\ee}{\mathrm{ee}}
\newcommand{\II}{\mathrm{II}}
\renewcommand{\c}{\mathrm{c}}
\newcommand{\x}{\mathrm{x}}
\newcommand{\Hxc}{\mathrm{Hxc}}
\newcommand{\xc}{\mathrm{xc}}
\newcommand{\eff}{\mathrm{eff}}


\newtheorem{thm}{\protect\theoremname}
\theoremstyle{plain}
\newtheorem{lem}[thm]{\protect\lemmaname}
\theoremstyle{remark}
\newtheorem{rem}[thm]{\protect\remarkname}
\theoremstyle{plain}
\newtheorem*{lem*}{\protect\lemmaname}
\theoremstyle{plain}
\newtheorem{prop}[thm]{\protect\propositionname}
\theoremstyle{plain}
\newtheorem{cor}[thm]{\protect\corollaryname}
\newtheorem{assumption}[thm]{Assumption}
\newtheorem{defn}[thm]{Definition}
\newtheorem{notation}[thm]{Notation}
\newtheorem{fact}[thm]{Fact}
\providecommand{\corollaryname}{Corollary}
\providecommand{\lemmaname}{Lemma}
\providecommand{\propositionname}{Proposition}
\providecommand{\remarkname}{Remark}
\providecommand{\theoremname}{Theorem}

\newcommand{\LL}[1]{\textcolor{blue}{[LL:#1 ]}}


\begin{document}

%+Title
\title{ScalES: Developer's manual}
\author{Authors}
\date{\today}
\maketitle


\section{Introduction}

The main goal of this developer's manual is to document and explain the implementation and design choices, that are neither simple to be documented directly in the code, nor suitable for publication in the literature.  

\section{Energy evaluation}

The Kohn-Sham total energy takes the form
\begin{equation}
    \begin{aligned}
    E^{\text{KS}}=& \frac{1}{2} \sum_{i=1}^{N} \int \left|\nabla \psi_{i}(\vr)\right|^{2} \ud \vr +\sum_{I} \int V_{\text{loc}, I}\left(\vr-\vR_{I}\right) \rho(\vr) \ud \vr +\frac{1}{2} \iint v_C(\vr,\vr')\rho(\vr) \rho(\vr') \ud \vr \ud \vr'\\
    &+\int E_{\text{xc}}[\rho(\vr)]\ud \vr+\sum_{I} \sum_{i=1}^{N} \sum_{l=1}^{L} \gamma_{I, \ell}\left(\int \psi_{i}(\vr) b_{I, \ell}\left(\vr-\vR_{I}\right)\right)^{2} +E_{\II}. 
    \end{aligned}
\label{eqn:KSEnergy}
\end{equation}
Here $V_{\text{loc},I}$ is the local component of the pseudopotential from the $I$-th atom, $\gamma_{I,\ell},b_{I,\ell}$ are the $\ell$-th nonlocal pseudopotential of the $I$-th atom, and $v_C$ is the Coulomb kernel, and sometimes we just write
\begin{equation}
v_C(\vr,\vr')=\frac{1}{|\vr-\vr'|}
\label{eqn:vC}
\end{equation}
even when periodic boundary conditions are used. 
We may also write the nonlocal pseudopotential as
\begin{equation}
V_{\text{nl}}(\vr,\vr';\{\vR_{I}\})=\sum_{I=1}^{M}\sum_{\ell=1}^{L_{I}} \gamma_{\ell,I}
  b_{\ell,I}(\vr-\vR_{I}) b_{\ell,I}^{*}(\vr'-\vR_{I}).
\label{eqn:nonlocal}
\end{equation}

We have omitted the integration domain $\Omega$ (in PWDFT / ScalES we always assume the periodic boundary condition), and omitted the technical issues associated with the periodic boundary conditions (such as Ewald summation, which will be properly taken into account in the pseudocharge formulation). So we may write the ion-ion interaction term as

\begin{equation}
E_{\II}=\frac12 \sum_{I\ne J} Z_I Z_J v_C(\vR_I,\vR_J).
\label{eqn:ionionenergy}
\end{equation}

In the diagonalization procedure, the total energy can also be evaluated as

\begin{equation}
  E^{\text{KS}}=\sum_{i=1}^N \varepsilon_i - \frac{1}{2} \iint v_C(\vr,\vr')\rho(\vr) \rho(\vr') \ud \vr \ud \vr-\int
    \rho(\vr)V_{\xc}[\rho](\vr) \ud \vr+E_{\xc}[\rho] + E_{\II}.
  \label{eqn:KSEnergy2}
\end{equation}

 
\section{Force evaluation}\label{sec:force}

This section documents the evaluation of the Hellmann-Feynman force in PWDFT / ScalES with the ONCV pseudopotential, as well as the rationale for choosing certain representation over others. In particular, we explain the ways to evaluate the local component of the force using the technique of pseudocharge / local pseudopotential + compensation charge, and the evaluation of the nonlocal component of the force using the density matrix formulation.

Once the SCF iteration
reaches convergence, the force on the $I$-th atom can be
computed as the negative derivative of the total energy with respect to
the atomic position $\vR_{I}$:
\begin{equation}
  \vF_{I} = -\frac{\partial E^{\text{KS}}(\{\vR_{I}\})}{\partial
  \vR_{I}}.
  \label{eqn:forceDef}
\end{equation}
The cost of the force calculation is
greatly reduced via the Hellmann-Feynman theorem,
which states that, at self-consistency,
the partial derivative $\frac{\partial}{\partial \vR_{I}}$ only needs to be applied to
terms in \cref{eqn:KSEnergy} which depend \textit{explicitly} on the
atomic position $\vR_{I}$. The Hellmann-Feynman (HF) force is then given by
\begin{equation}
  \begin{split}
    \vF_{I} = &-\int \frac{\partial
    V_{\text{loc}}}{\partial \vR_{I}}(\vr;\{\vR_{I}\}) \rho(\vr) \ud \vr
    - \sum_{i=1}^{N} \int \psi_{i}^{*}(\vr)
    \frac{\partial V_{\text{nl}}}{\partial \vR_{I}}(\vr,\vr';\{\vR_{I}\})
    \psi_{i}(\vr') \ud \vr \ud \vr'\\
    &-\frac{\partial E_{\II}(\{\vR_{I}\})}{\partial
  \vR_{I}}.
  \end{split}
  \label{eqn:forceHF1}
\end{equation}

Using integration by parts, the force can be rewritten as
\begin{equation}
  \begin{split}
    \vF_{I} = &\int \nabla_{\vr} V_{\text{loc},I}(\vr-\vR_{I})
    \rho(\vr) \ud \vr\\
    & + 2 \Re \sum_{i=1}^{N} \sum_{\ell=1}^{L_{I}} \gamma_{I,\ell}
    \left(\int \psi^{*}_{i}(\vr) \nabla_{\vr} b_{I,\ell}(\vr-\vR_{I}) \ud \vr \right)
    \left(\int b^{*}_{I,\ell}(\vr'-\vR_{I}) \psi_{i}(\vr') \ud \vr' \right)\\
    & -\frac{\partial E_{\II}(\{\vR_{I}\})}{\partial
  \vR_{I}}.
  \end{split}
  \label{eqn:forceHF2}
\end{equation}
We can further split the force into the local component,
\begin{equation}
 \vF_{\text{loc},I}=\int \nabla_{\vr} V_{\text{loc},I}(\vr-\vR_{I})
    \rho(\vr) \ud \vr-\frac{\partial E_{\II}(\{\vR_{I}\})}{\partial
  \vR_{I}}.
\label{eqn:force_loc}
\end{equation}
and the nonlocal component
\begin{equation}
 \vF_{\text{nl},I}= 2 \Re \sum_{i=1}^{N} \sum_{\ell=1}^{L_{I}} \gamma_{I,\ell}
    \left(\int \psi^{*}_{i}(\vr) \nabla_{\vr} b_{I,\ell}(\vr-\vR_{I}) \ud \vr \right)
    \left(\int b^{*}_{I,\ell}(\vr'-\vR_{I}) \psi_{i}(\vr') \ud \vr' \right).
\label{eqn:force_nl}
\end{equation}

\subsection{Local component}

The local pseudopotential $V_{\text{loc},I}(\vr)$ decays as $1/|\vr|$ as $|\vr|\to \infty$ and so is the ion-ion interaction. Therefore the direct evaluation of \cref{eqn:forceHF2} would be too costly. There are two ways to overcome this issue. The first is the \textit{pseudocharge formulation}. Define
\begin{equation}
V_{\text{loc},I}(\vr)=\int v_C(\vr,\vr') \rho_{\text{loc},I}(\vr')\ud \vr,
\label{eqn:pseudocharge}
\end{equation}
where $\rho_{\text{loc},I}$ is the pseudocharge and is localized in the real space. Assuming the pseudocharges from different atoms do not overlap (this turns out to be a severe restriction when atoms are close to each other, and corrections need to be included when this is violated, see \cite{PaskSterne2005,SolerArtachoGaleEtAl2002}), then the ion-ion interaction energy can be written (using Gauss's law) as 
\begin{equation}
\begin{aligned}
E_{\II}=&\frac12 \iint v_C(\vr,\vr') \rho_{\text{loc}}(\vr)\rho_{\text{loc}}(\vr') \ud \vr \ud \vr' -\frac12  \sum_{I} \iint v_C(\vr,\vr') \rho_{\text{loc},I}(\vr)\rho_{\text{loc},I}(\vr') \ud \vr \ud \vr'\\
=&\frac12  \iint v_C(\vr,\vr') \rho_{\text{loc}}(\vr)\rho_{\text{loc}}(\vr') \ud \vr \ud \vr' - \frac12 \sum_{I} \int V_{\text{loc},I}(\vr) \rho_{\text{loc},I}(\vr) \ud \vr.
\end{aligned}
\label{eqn:EII_pseudocharge}
\end{equation}
The second term in \cref{eqn:EII_pseudocharge} is a constant and is independent of the atomic position. So it does not contribute to the force. The contribution from the first term is
\[
 -\frac{\partial E_{\II}(\{\vR_{I}\})}{\partial
  \vR_{I}}=\int V_{\text{loc}}(\vr) \nabla_\vr \rho_{\text{loc},I}(\vr) \ud \vr=-\int  \nabla_\vr V_{\text{loc}}(\vr) \rho_{\text{loc},I}(\vr) \ud \vr.
\]
Here we have used integration by parts as well as the periodic boundary condition. Then using integration by parts again, the contribution to the local component of the force is
\begin{equation}
\begin{aligned}
\vF_{\text{loc},I}=&\int \nabla_{\vr} \rho_{\text{loc},I}(\vr-\vR_{I})
    V_{\text{H}}(\vr) \ud \vr-\int  \nabla_\vr V_{\text{loc}}(\vr) \rho_{\text{loc},I}(\vr) \ud \vr\\
=&-
    \int  \nabla_\vr (V_{\text{H}}(\vr)+V_{\text{loc}}(\vr)) \rho_{\text{loc},I}(\vr) \ud \vr
\end{aligned}
\label{eqn:force_loc2}
\end{equation}
Note that although $V_{\text{H}}(\vr),V_{\text{loc}}(\vr)$ are themselves difficult to evaluate due to the long range interaction, \textit{for charge neural systems},  their sum
\[
V_{\text{H}}(\vr)+V_{\text{loc}}(\vr)=\int v_C(\vr,\vr') (\rho(\vr')+\rho_{\text{loc}}(\vr'))\ud \vr'
\]
can be conveniently evaluated by solving a Poisson's equation with periodic boundary conditions using FFT. 

\begin{rem}
There are two considerations in writing the formula above. The first is to avoid long range summation. The second is to avoid the explicit computation of $\nabla_\vr \rho_{\text{loc},I}$. This quantity is the third order derivative of $V_{\text{loc},I}$ which is actually provided by most pseudopotentials formats, and can be highly oscillatory.
In fact, for many pseudopotentials (such as Troullier-Martins, and even the ONCV pseudopotential), even $\rho_{\text{loc},I}(\vr)$ is too oscillatory and cannot be directly used in a real space based code. Therefore one has to use the original $V_{\text{loc},I}$. This leads to the \texttt{Use\_VLocal} option in ScalES. Therefore when TM and ONCV potentials are used, \texttt{Use\_VLocal} must be set to \texttt{true}.
\end{rem}

\begin{rem}
In ScalES, the electron density $\rho(\vr)$ is assumed to be positive, and hence the ionic charge must be negative. For legacy reason, $\rho_{\text{loc}}$ is stored with a  \textit{positive} total charge (the pointwise value can still be negative depending on the shape of the potential). So in the implementation, $\rho_{\text{loc}}$ should be interpreted as $-\rho_{\text{loc}}$. 
\end{rem}

When \texttt{Use\_VLocal==true}, we introduce the so called compensation charge to handle the long range interaction, denoted by $\rho_{\text{cp},I}$. The compensation charge is chosen to be a Gaussian function, so that its corresponding $V_{\text{cp},I}$ can be evaluated analytically. The $\sigma$ value of the Gaussian function is small enough so that the overlap of the compensation charges between any two atoms is negligible. Then 
\[
V_{\text{sr},I}(\vr)=V_{\text{loc},I}(\vr)-V_{\text{cp},I}(\vr)
\] 
becomes a short range potential that can be stored as a sparse vector. Clearly when $\rho_{\text{cp},I}=\rho_{\text{loc},I}$, the short range potential vanishes. Following the same derivation above, we find that the local component of the force is
\begin{equation}
\vF_{\text{loc},I}=- \int  \nabla_\vr \rho(\vr) V_{\text{sr},I}(\vr) \ud \vr-
    \int  \nabla_\vr (V_{\text{H}}(\vr)+V_{\text{cp}}(\vr)) \rho_{\text{loc},I}(\vr) \ud \vr,
\label{eqn:force_loc3}
\end{equation}
where $V_{\text{cp}}(\vr)=\sum_{I}V_{\text{cp},I}(\vr)$. This is the formulation used in PWDFT and ScalES. 

\begin{rem}
In \cref{eqn:force_loc3}, we again choose not to compute the gradient for $V_{\text{sr},I}(\vr)$, but use integration by parts to move it to $\rho$, which is observed to be smoother. 
\end{rem}

\subsection{Nonlocal component}

In ScalES, the efficient way to evaluate the nonlocal contribution is through the reduced density matrix. This is documented in \cite[Eq. (29)]{ZhangLinHuEtAl2017}.

\section{Programming}

\bibliographystyle{abbrv}
\bibliography{developer}

\end{document}



